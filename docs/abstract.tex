%%%%%%%%%%%%%%%%%%%%%%%%%%%%%%%%%%%%%%%%%%%%%%%%%%%%%%%%%%%%%%%%%%%%%%%%%%%%%%%%
% ABSTRACT


\begin{abstract} %% insira abaixo seu abstract

\hypertarget{estilo:abstract}{} %% uso para este Guia

Recently, several evolutionary computation techniques have been used in research areas such as parameter estimation of linear and nonlinear dynamic processes. This motivates the use of algorithms such as the particle swarm optimization (PSO) in the aforementioned fields of knowledge. However, little is known about the convergence of this algorithm and mainly the analyzes and studies have focused on experimental results. Therefore, the objective of this work is to propose a structure for the PSO that better analyze the convergence of the algorithm analytically. For this, the PSO is restructured to assume a matrix form, reformulated as a piecewise linear system. There was a convergence analysis of the algorithm as a whole, using an almost sure convergence criterion applicable to switched systems. Subsequently, traditional parameter identification algorithms were combined with the matricial PSO (MPSO), so as to make the identification results as good as or better than identifying only using the PSO or only the traditional algorithms. The obtained functions, after the identification, using the matricial PSO algorithm combined with the conventional identification algorithms, presented a better generalization and proper identification. The conclusions reached were that the hybridization permits a minimum performance and also contributes to improve the results obtained with the traditional algorithms, allowing the system representation in a higher range of frequencies.

\vspace{24pt}

\noindent{\bf Keywords}:  PSO, Convergence Analysis, Identification.
\end{abstract}


