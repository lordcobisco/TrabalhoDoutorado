%%%%%%%%%%%%%%%%%%%%%%%%%%%%%%%%%%%%%%%%%%%%%%%%%%%%%%
%Anexos
\hypertarget{estilo:anexo}{} %% uso para este manual

%%%%%%%%%%%%%%%%%%%%%%%%%%%%%%%%%%%%%%%%%%%%%%%%%%%%%%%%%%%%%%%%%%%%%%%%%%%%%%%%%
\chapter{AP�NDICE - EQUA��ES QUE RELACIONAM AS ENTRADAS E SA�DAS DOS HIDROCICLONES} %% 
\label{APENDICE_B} %% 

A descri��o do modelo matem�tico hidrodin�mico do hidrociclone se inicia a partir das seguintes equa��es:

\begin{equation}
\begin{array}{rcll}
\Delta P_0 = & \alpha_1 W_0
\end{array}
\label{eq:VarPressaoZero}
\end{equation}

\begin{equation}
\begin{array}{rcll}
\Delta P_u = & \alpha_2 W_u
\end{array}
\label{eq:VarPressaoU}
\end{equation}

Onde:

$W_0$ - vaz�o na linha superior de topo;

$W_u$ - vaz�o na linha de fundo do hidrociclone;

$\alpha_1$ e $\alpha_2$ - par�metros de ajuste do modelo para perda de carga;

$\Delta P_0$ - diferen�a de press�o entre a corrente de alimenta��o e a corrente de descarga superior do hidrociclone;

$\Delta P_u$ - diferen�a de press�o entre a corrente de alimenta��o e a corrente de descarga inferior do hidrociclone.\\


Resolvendo o sistema de equa��es mostrado em \ref{eq:VarPressaoZero} e \ref{eq:VarPressaoU} � poss�vel encontrar a seguinte rela��o que envolve as vaz�es nas linhas superior e inferior do hidrociclone como pode ser visto nas equa��es \ref{eq:WZero} e \ref{eq:WU}.

\begin{equation}
\begin{array}{rcll}
W_0 = & \frac{Cv_{max,0} . S_0}{0,0693   60   \rho_{fl}} \sqrt{ d_l (P_1 - \Delta P_0 - P_0)}
\end{array}
\label{eq:WZero}
\end{equation}

\begin{equation}
\begin{array}{rcll}
W_u = & \frac{Cv_{max,u} . S_u}{0,0693 . 60  . \rho_{fw}} \sqrt{ d_w (P_1 - \Delta P_u - P_u)}
\end{array}
\label{eq:WU}
\end{equation}

Onde:

$Cv_{max,0}$ - coeficiente de descarga m�ximo da v�lvula de topo;

$Cv_{max,u}$ - coeficiente de descarga m�ximo da v�lvula de fundo;

$d_l$ - densidade espec�fica do �leo;

$d_w$ - densidade espec�fica da �gua;

$P_0$ - press�o na descarga da linha de topo;

$P_u$ - press�o na descarga da linha de fundo;

$P_1$ - press�o na alimenta��o do hidrociclone;

$S_o$ - abertura da v�lvula de topo;

$S_u$ - abertura da v�lvula de fundo;

$\rho_{fl}$ - massa espec�fica da fase oleosa;

$\rho_{fw}$ - massa espec�fica da fase aquosa.
